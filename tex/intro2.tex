%%%%%%%%%%%%%%%%%%%%%% Intro %%%%%%%%%%%%%%%%%%%%%%%%%%%%%

\section{Introducci\'on}
\label{sec:intro}
Los modelos bidimensionales han tenido un importante papel en la mec\'anica estad\'istica en relaci\'on al estudio de los
 fen\'omenos cr\'iticos y las transiciones de fase. Entre los más estudiados se encuentran los que representan
 varios tipos de spines dispuestos en una red vinculados por algún tipo de interacción, por ejemplo el modelo de Ising, Eigth-Vertex (8V)\cite{model_8V},
 o Ashkin-Teller (AT)\cite{ashkin_teller_43}. Este tipo de modelos ha sido objeto de diversos estudios durante las últimas décadas,
 algunos de ellos han sido resueltos exactamente a través de sofisticados métodos teóricos y estudiados mediante varios
 métodos numéricos, por lo que existe abundante información sobre su comportamiento. Sin embargo, la posibilidad de obtener y estudiar en un laboratorio
 materiales bidimensionales \cite{exp_ultrathin_magfilms} ha despertado interés en el estudio de nuevos problemas y nuevos escenarios,
 entre ellos, la introducci\'on de interfaces y defectos topol\'ogicos en los modelos mencionados \cite{linear_defects2D, interf, interf_AT}
 con el objetivo de describir tanto la presencia de impurezas en los materiales como las modificaciones realizadas a estos en diferentes procesos
 de fabricaci\'on. Estos avances han conducido al descubrimiento de importantes modificaciones en el comportamiento cr\'itico de los sistemas.\\
El modelo de Ashkin Teller, resulta de particular inter\'es debido a la riqueza de su comportamiento cr\'itico y su generalidad, ya que puede relacionarse con
 varios modelos bidimensionales a trav\'es de transformaciones [buscar referencia mapping]. Puede ser considerado como dos sistemas de Ising[ref]
 bidimensionales acoplados por una interacci\'on entre cuatro spines (2 de cada plano).\\

El comportamiento crítico del modelo de Ashkin-Teller en dos dimensiones con un defecto en forma de línea ha sido estudiado
 recientemente mediante diferentes m\'etodos, y en particular se han calculado sus exponentes críticos. Na\'on \cite{AT_naon} ha introducido un defecto asim\'etrico, mediante la modificaci\'on
 de los acoplamientos en solo uno de los planos de spines, y estudiado el comportamiento de la funci\'on de correlaci\'on
 spin-spin sobre la línea defectuosa mediante el método de integrales de camino en la descripción del modelo en términos de campos fermiónicos.
 Lajk\'o e Igl\'oi \cite{AT_lajko}
 han utilizando el método numérico del grupo de renormalización de la matriz densidad (DMRG) en la descripción del modelo en el límite Hamiltoniano,
 que conduce a una representación cuántica en una red de dimensión menor a la del modelo clásico. Ambos trabajos se refieren al estudio de
 la dependencia de los exponentes críticos del sistema con algunos parámetros del Hamiltoniano presentando discrepancias, en particular, en los resultados
 obtenidos para el exponente crítico de la magnetización del defecto. Mientras que en la Ref. \cite{AT_naon} se encuentra que la interacci\'on entre ambos subsistemas tipo Ising
 no afecta a los exponentes cr\'iticos, los resultados presentados en la Ref. \cite{AT_lajko} muestran una dependencia expl\'icita del exponente cr\'itico de la magnetizaci\'on
 sobre el defecto con el acoplamiento entre ambos modelos de Ising. Estas discrepancias son una fuerte motivación para estudiar el sistema con defectos utilizando
 un método diferente.\\ 
En este trabajo estudiaremos el modelo de Ashkin Teller utilizando el método de simulaciones computacionales Montecarlo (MC), ampliamente utilizado
 para el estudio de fenómenos críticos en sistemas bidimensionales, aplicado a la representación clásica del modelo utilizando dos sistemas bidimensionales de spines,
 con interacci\'on a primeros vecinos (modelo de Ising) acoplados entre s\'i mediante una interacci\'on de cuatro spines. Comprobaremos el comportamiento general de nuestros
 algoritmos determinando de manera cualitativs el diagrama de fases del modelo. Luego introduciremos un defecto asim\'etrico con forma de l\'inea y estudiaremos el comportamiento
 cr\'itico del sistema con AT con defecto, poniendo inter\'es en particular en la determinaci\'on numérica de la dependencia del exponente crítico asociado a la magnetización
 local del defecto con las constantes de acoplamiento.\\

%%%%%%%%%%%%%%%%% Aspectos teóricos %%%%%%%%%%%%%%%%%%%%%%%%

\section{Aspectos teóricos del comportamiento crítico}
\label{sec:teoria}

En esta sección haremos un breve resumen de deficiones y resultados derivados de
 diferentes teorías utilizadas en el análisis del comportamiento crítico de
 sistemas termodinámicos.

\subsection{Exponentes críticos y parámetros de orden}

El término fenómenos críticos se refiere a las propiedades termodinámicas de
sistemas físicos cerca de la temperatura crítica $T_{c}$ de una transición de
fase. Los exponentes críticos describen el comportamiento de magnitudes
 termodinámicas medibles cerca del punto crítico, suponiendo que pueden
 descomponerse en una parte regular que mantiene un valor finito y una parte
 singular que puede ser divergente o tener derivadas divergentes.\\

Si una magnitud física de interés puede ser descripta por una función $F(T, V, P)$
 de las variables que definen el estado termodinámico de un sistema, y
 consideramos el parámetro $t=\frac{T-T_{c}}{T_{c}}$ que mide la desviación de
 la temperatura respecto al valor crítico $T_{c}$, el exponente crítico $\lambda${}
 asociado a $F(t)$ está definido por la relación:
\\
\begin{equation}
	\lambda\equiv \lim_{t \to 0}\frac{\ln{F(t)}}{\ln{t}},
	\label{eq:defexpcrit}
\end{equation}
 que implica que la magnitud descripta por $F$ se comporta como una ley de
 potencias en la región crítica:
\begin{equation}
	F(t)\sim t^{\lambda}
	\label{eq:leypot}
\end{equation}
\\
El sistema bajo estudio en este trabajo puede expresarse en los mismos términos
 que el modelo de Ising bidimensional (diremos más sobre esto en las siguientes
 secciones), por ello las magnitudes físicas que consideraremos en su estudio
 son las utilizadas en el caso de los sistemas magnéticos o sistemas formados por spines,
 la magnetización $M$, la susceptibilidad $\chi$ y la capacidad calorífica $C$. Los exponentes
 críticos asociados están definidos según las siguientes relaciones:
\\
\begin{center} 
\begin{eqnarray}
	\label{eq:expcrit_mag}
	M\sim \abs{t}^{\beta} \\
	\label{eq:expcrit_susc}
	\chi\sim \abs{t}^{-\gamma} \\
	\label{eq:expcrit_cap}
	C\sim \abs{t}^{-\alpha} \\
\end{eqnarray}
\end{center}
Otra magnitud importante en el estudio de sistemas compuestos por spines es la
 función de correlación $G(r, r')$, que mide la probabilidad de que los spines en las
  posiciones $r$ y $r'$ se encuentren en el mismo estado y representa por lo tanto
  una medida del orden espacial del sistema. La función de correlación para un sistema
  compuesto por spines $\sigma$ en una red bidimensional está dada por:

\begin{equation}
	G(ij, km) = \mean{(\sigma_{ij}-\mean{\sigma_{ij}})(\sigma_{km}-\mean{\sigma_{km}})} 
	\label{eq:2dcorrfunct}
\end{equation}
donde los subíndinces $ij$ y $km$ denotan la posición de los spines en la red bidimensional.
Su comportamiento cerca del punto crítico puede ser descripto en forma de ley de potencias,
 introduciendo la cantidad $xi$, llamada longitud de correlación, y el exponente crítico $p$:
\begin{center} 
	\begin{eqnarray}
		\label{eq:expcrit_funccorr}
		G(r) = r^{-p}e^{\frac{-r}{\eta}}\\
		%G(r)\sim r^{-(d-2+\eta)}
		\label{eq:expcrit_longcorr}
		\xi\sim \abs{t}^{-\nu}
	\end{eqnarray}
\end{center}
donde $p=d-2+\eta$ es el exponente crítico asociado a la función de correlación y
$\nu$ el asociado a la longitud de correlación. Dado que esta última magnitud
diverge en el punto crítico, la expresión para $G(r)$ cuando $t\rightarrow 0$
resulta:
\begin{equation}
	G(r)\sim r^{-(d-2+\eta)}
	\label{eq:corrfunct}
\end{equation}
%\begin{center} 
%\begin{eqnarray}
%	\label{eq:op_Ms}
%	\mean{\sigma}&=&\frac{1}{L^{2}}\sum_{ij}\sigma_{ij} \\
%	\label{eq:op_Mt}
%	\mean{\tau}&=&\frac{1}{L^{2}}\sum_{ij}\tau_{ij} \\
%	\label{eq:op_Mst}
%	\mean{\sigma\tau}&=&\frac{1}{L^{2}}\sum_{ij}\sigma_{ij}\tau_{ij} \\
%	\label{eq:op_stagMs}
%	\mean{\sigma}_{AF}&=&\frac{1}{L^{2}}\sum_{ij}(-1)^{(i+j)}\sigma_{ij} \\
%	\label{eq:op_stagMt}
%	\mean{\tau}_{AF}&=&\frac{1}{L^{2}}\sum_{ij}(-1)^{(i+j)}\tau_{ij} \\
%	\label{eq:op_stagMst}
%	\mean{\sigma\tau}_{AF}&=&\frac{1}{L^{2}}\sum_{ij}(-1)^{(i+j)}\sigma_{ij}\tau_{ij}
%\end{eqnarray}
%\end{center}

Una transición de fase  del tipo orden-desorden puede caracterizarse cuantitativamente
 por alguna cantidad, usualmente una variable termodinámica medible, que toma un valor
 nulo en la fase desordenada y un valor diferente de cero en la fase ordenada.
 Dicha cantidad es llamada parámetro de orden y su definición, en términos de
 magnitudes físicas, depende del sistema bajo estudio. Por ejemplo en el caso
 de un sistema magnético, la magnetización espontánea $M$ (el promedio de los momentos magnéticos de los
 elementos que componen el sistema) es una medida del orden ferromagnético del
 sistema.\\
El modelo que estudiaremos en este trabajo describe un sistema conformado por spines
 que pueden tomar dos valores posibles ubicados en una red bidimensional cuadrada.
Definimos aquí los parámetros que nos servirán para caraterizar los diferentes
 estados de orden en el sistema:

\begin{center} 
\begin{eqnarray}
	\label{eq:op_Ms}
	\mean{\sigma}&=&\frac{1}{L^{2}}\sum_{ij}\sigma_{ij} \\
	\label{eq:op_stagMs}
	\mean{\sigma}_{AF}&=&\frac{1}{L^{2}}\sum_{ij}(-1)^{(i+j)}\sigma_{ij} \\
\end{eqnarray}
\end{center}
el subíndice $ij$ en estas ecuaciones indica la posición del spin $\sigma$ en una red bidimensional cuadrada de lado $L$.
El parámetro definido en la ec. (\ref{eq:op_Ms}) representa la magnetización media, medida del orden ferromagnético, de un sistema compuesto por spines $\sigma$,
 toma valores no nulos cuando más de la mitad de los spines se encuentran alineados.
 El definido por la ec. (\ref{eq:op_stagMs}) es una medida del orden de tipo antiferromagnético y su valor indica la proporción de
 spines alineados de forma alternada. En general cuando alguno de estos parámetros es no nulo el sistema se encuentra en un estado ordenado.\\

\subsection{Teoría de escala}
\label{sec:teoria_escala}
Según la hipótesis de escala \cite{fisher_scaling} la parte singular de la
 energía libre $f$ de un sistema es una función homogénea generalizada de los
 parámetros que miden la desviación respecto al punto crítico (como el parámetro $t$
 definido en la sección anterior o $h=\frac{H-H_{c}}{H_{c}}$ definido de manera 
 equivalente para el campo magnético, con $H_{c}$ el valor crítico del campo magnético)
 que, ante un cambio de escala en las variables $r\rightarrow r/b$, transforma como:
\begin{equation}
	\label{eq:sca_hyp}
	f(t,h,\frac{1}{L})=b^{-d}f(b^{1/\nu}t,b^{d-x}h,\frac{b}{L})
\end{equation}
\\

A partir de este comportamiento y las relaciones entre las magnitudes termodinámicas
 y las derivadas de $f$ pueden obtenerse relaciones entre los exponentes críticos
 definidos en las ecs. (\ref{eq:expcrit_mag}-\ref{eq:expcrit_funccorr}).
 Por ejemplo la magnetización del
 sistema $M=-\partial f/\partial h$ transforma como:

\begin{equation}
	\label{eq:sca_mag}
	M(t,h,\frac{1}{L})=b^{-x}M(b^{1/\nu}t,b^{d-x}h,\frac{b}{L})
\end{equation}
\\
Haciendo uso de la arbitrariedad del parámetro $b$, que define a la transformación
de escala, y considerando el límite $L\rightarrow \infty$, para $h=0$, puede elegirse
 $b=L$ obteniéndose:

%\begin{center} 
%	\begin{eqnarray}
\begin{equation}
		\label{eq:xdefinition}
		M\sim L^{-x}
\end{equation}
%	\end{eqnarray}
%\end{center}
que relaciona la magnetización del sistema con el tamaño del mismo a través de
 una ley de potencias en las cercanías del punto crítico.
Si en cambio se elige $b=t^{-\nu}$ resulta:

\begin{equation}
	\label{eq:sca_rel_1}
	\beta =x\nu
\end{equation}
 una relación entre algunos delos exponentes críticos del sistema.\\
Utilizando estos mismos argumentos junto con las definiciones de la capacidad
calorífica $C=\partial^{2} f/\partial t^{2}$ y la susceptibilidad
 $\chi=\partial^{2} f/\partial h^{2}$ pueden obtenerse otras relaciones entre
 los exponentes críticos:

\begin{center} 
\begin{eqnarray}
	\label{eq:sca_rel_2}
	d\nu =2\beta +\gamma \\
	\label{eq:sca_rel_3}
	2-\eta =\frac{\gamma}{\nu} \\
	\label{eq:sca_rel_4}
	\alpha = 2 - d\nu
\end{eqnarray}
\end{center}

Cuando se desea describir un sistema espacialmente acotado, como la presencia de
 un borde, un defecto o la unión entre dos materiales diferentes, el modelo que
 describe el sistema se debe tenerse en cuenta, en las zonas en las que el sistema
 no presenta homogeneidad espacial, la contribución local a la energía libre.
 Suponiendo que la parte singular de la energía libre asociada a esta contribución
 $f_{def}$ presenta el mismo comportamiento que el dado en la ec. (\ref{eq:sca_hyp})
 para $f$ y considerando el caso de un defecto con dimensión $d-1$, por ejemplo
 una superficie en un sistema tridimensional o un defecto lineal en uno bidimensional:
 
\begin{equation}
	\label{eq:sca_def}
	f_{def}(t,h_{def},\frac{1}{L})=b^{-(d-1)}f_{def}(b^{1/\nu}t,b^{d-x}h_{def},\frac{b}{L})
\end{equation}
\\
 donde $h_{def}$ representa el campo magnético sobre la superficie o defecto.\\
Repitiendo los argumentos utilizados anteriormente, las relaciones \ref{eq:xdefinition}-\ref{eq:sca_rel_1} se cumplen para
 los correspondientes exponentes de superfice (defecto):
 
\begin{equation}
	\label{eq:sca_rel_surf}
	M_{def}\sim L^{-x_{def}}, \beta_{def} =x_{def}\nu
\end{equation}
\\
La funci\'on de correlaci\'on para un par de spines se obtiene como la derivada funcional de la energ\'ia libre respecto del campo magn\'etico 
 $G(r,t)=\mean{\sigma(0)\sigma(r)}=\frac{\delta F}{\delta h(0) \delta h(r)}$ y por lo tanto su comportamiento ante la
 transformaci\'on de escala $r\rightarrow r/b$ viene dado por:
 
\begin{equation}
	\label{eq:sca_corr}
	G(r, t)=b^{-2x}G(\frac{b}{r}, b^{1/\nu}t)
\end{equation}
\\
 en las cercan\'ias del punto cr\'itico, $t\rightarrow 0$, y para una elecci\'on de $b=r$ las correlaciones decaen como
 una ley de potencias:

\begin{equation}
	\label{eq:sca_corr_pot}
	G(r, t=0)=\frac{G_{0}}{r^{2x}}
\end{equation}
\\
En presencia de defectos, pueden analizarse las correlaciones entre spines pertenecientes al defecto, y las
 correlaciones entre estos y los que no pertenecen al defecto.\\
En el caso de un defecto en forma de l\'inea las primeras
 son correlaciones a lo largo de la l\'inea, longitudinales, y las \'ultimas son perpendiculares al defecto. En ambos casos
 el exponente de la funci\'on de correlaci\'on est\'a relacionado con el exponente cr\'itico $x_{def}$ de la magnetizaci\'on
 del defecto definido en \ref{eq:sca_rel_surf}:

\begin{equation}
	\label{eq:sca_corr_pot_def}
	G_{\parallel}(r, t=0)\sim{r^{-2x_{def}}}, \; \; \; \;  G_{\perp}(r, t=0)\sim{r^{-(x+x_{def})}}
\end{equation}
\\

Estas expresiones nos permiten estudiar el comportamiento cr\'itico de la funci\'on de correlaci\'on a partir
 de medidas realizadas para el exponente cr\'itico $x_{def}$ asociado a la magnetizaci\'on del defecto.\\

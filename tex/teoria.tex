%%%%%%%%%%%%%%%%% Aspectos teóricos %%%%%%%%%%%%%%%%%%%%%%%%

\section{Aspectos teóricos del comportamiento crítico}
\label{sec:teoria}

En esta sección haremos un breve resumen de deficiones y resultados derivados de diferentes teorías utilizadas en el análisis del comportamiento crítico de sistemas
 termodinámicos.\\

\subsection{Exponentes críticos y parámetros de orden}

Los exponentes críticos describen el comportamiento de varias magnitudes termodinámicas medibles cerca del punto crítico, suponiendo que pueden descomponerse en
 una parte regular que mantiene un valor finito y una parte singular que puede ser divergente o tener derivadas divergentes.\\
Considerando el parámetro $t=\frac{T-T_{c}}{T_{c}}$ que mide la desviación de la temperatura respecto al valor crítico $T_{c}$,
 se define el exponente crítico $\lambda$ asociado a la función $F(t)$, que describe el comportamiento de alguna magnitud
 f\'isica de inter\'es en t\'erminos de variables termodin\'amicas:
\\
\begin{equation}
	\lambda\equiv \lim_{t \to 0}\frac{\ln{F(t)}}{\ln{t}},
	\label{eq:defexpcrit}
\end{equation}
\\
 generalmente la relación entre $F(t)$ y $\lambda$ se denota como $F(t)\sim t^{\lambda}$.\\
En el caso de un sistema magnético los exponentes críticos para la magnetización $M$, la susceptibilidad $\chi$, la capacidad calorífica $C$, la longitud de correlación
 $\xi$ y la función de correlación $G(r)$ están definidos como:
\\
\begin{center} 
\begin{eqnarray}
	\label{eq:expcrit_mag}
	M\sim \abs{t}^{\beta} \\
	\label{eq:expcrit_susc}
	\chi\sim \abs{t}^{-\gamma} \\
	\label{eq:expcrit_cap}
	C\sim \abs{t}^{-\alpha} \\
	\label{eq:expcrit_longcorr}
	\xi\sim \abs{t}^{-\nu} \\
	\label{eq:expcrit_funccorr}
	G(r)\sim \abs{t}^{-(d-2+\eta)}
\end{eqnarray}
\end{center}

%\begin{center} 
%\begin{eqnarray}
%	\label{eq:op_Ms}
%	\mean{\sigma}&=&\frac{1}{L^{2}}\sum_{ij}\sigma_{ij} \\
%	\label{eq:op_Mt}
%	\mean{\tau}&=&\frac{1}{L^{2}}\sum_{ij}\tau_{ij} \\
%	\label{eq:op_Mst}
%	\mean{\sigma\tau}&=&\frac{1}{L^{2}}\sum_{ij}\sigma_{ij}\tau_{ij} \\
%	\label{eq:op_stagMs}
%	\mean{\sigma}_{AF}&=&\frac{1}{L^{2}}\sum_{ij}(-1)^{(i+j)}\sigma_{ij} \\
%	\label{eq:op_stagMt}
%	\mean{\tau}_{AF}&=&\frac{1}{L^{2}}\sum_{ij}(-1)^{(i+j)}\tau_{ij} \\
%	\label{eq:op_stagMst}
%	\mean{\sigma\tau}_{AF}&=&\frac{1}{L^{2}}\sum_{ij}(-1)^{(i+j)}\sigma_{ij}\tau_{ij}
%\end{eqnarray}
%\end{center}

Una transición de fase  del tipo orden-desorden puede ser cuantitativamente caracterizada por alguna cantidad que toma un valor nulo en la fase desordenada y un valor
diferente de cero en la fase ordenada. Dicha propiedad es llamada parámetro de orden y su definición en términos de magnitudes físicas depende del sistema
bajo estudio. Por ejemplo en el caso de un sistema magnético, la magnetización espontánea (el promedio de los momentos magnéticos de los elementos que componen
 el sistema) es una medida del orden ferromagnético del sistema.\\
A continuación se definen algunos parámetros que serán de utilidad en la descripción del modelo que estudiaremos:

\begin{center} 
\begin{eqnarray}
	\label{eq:op_Ms}
	\mean{\sigma}&=&\frac{1}{L^{2}}\sum_{ij}\sigma_{ij} \\
	\label{eq:op_stagMs}
	\mean{\sigma}_{AF}&=&\frac{1}{L^{2}}\sum_{ij}(-1)^{(i+j)}\sigma_{ij} \\
\end{eqnarray}
\end{center}

El parámetro definido en la ec. (\ref{eq:op_Ms}) representa la magnetización media, medida del orden ferromagnético, de un sitema compuesto por spines $\sigma$,
 toma valores no nulos cuando más de la mitad de los spines se encuentran alineados;
 el definido por la ec. la ec. (\ref{eq:op_stagMs}) es una medida del orden de tipo antiferromagnético y su valor indica la proporción de
 spines alineados de forma alternada. En general cuando alguno de estos parámetros es no nulo el sistema se encuentra en un estado ordenado.\\

\subsection{Teoría de escala}
\label{sec:teoria_escala}
Según la hipótesis de escala \cite{fisher_scaling} la parte singular de la energía libre $f$ de un sistema es una función homogénea generalizada de los
 parámetros que miden la desviación respecto al punto crítico (como $t$ o el parámetro $h=\frac{H-H_{c}}{H_{c}}$ para el campo magnético, con $H_{c}$ el valor crítico del campo magnético)
 que, ante un cambio de escala en las variables $r\rightarrow r/b$, transforma como:
\begin{equation}
	\label{eq:sca_hyp}
	f(t,h,\frac{1}{L})=b^{-d}f(b^{1/\nu}t,b^{d-x}h,\frac{b}{L})
\end{equation}
\\

A partir de este comportamiento y las relaciones entre las magnitudes termodinámicas y las derivadas de $f$ pueden obtenerse relaciones entre los exponente críticos
 definidos en las ecs. (\ref{eq:expcrit_mag}-\ref{eq:expcrit_funccorr}) haciendo uso de la arbitrariedad del parámetro $b$, así por ejemplo, la magnetización del
 sistema $M=-\partial f/\partial h$ transforma como:

\begin{equation}
	\label{eq:sca_mag}
	M(t,h,\frac{1}{L})=b^{-x}M(b^{1/\nu}t,b^{d-x}h,\frac{b}{L})
\end{equation}
\\
En el límite $L\rightarrow \infty$, para $h=0$ y eligiendo $b=L$ se obtiene que:

\begin{equation}
	\label{eq:xdefinition}
	M\sim L^{-x}
\end{equation}
\\
Si en cambio se elige $b=t^{-\nu}$ resulta:

\begin{equation}
	\label{eq:sca_rel_1}
	\beta =x\nu
\end{equation}
\\
Utilizando estos mismos argumentos junto con las definiciones de $C=\partial^{2} f/\partial t^{2}$ y $\chi=\partial^{2} f/\partial h^{2}$
 pueden obtenerse otras expresiones que relacionan los exponentes críticos:

\begin{center} 
\begin{eqnarray}
	\label{eq:sca_rel_2}
	d\nu =2\beta +\gamma \\
	\label{eq:sca_rel_3}
	2-\eta =\frac{\gamma}{\nu} \\
	\label{eq:sca_rel_4}
	\alpha = 2 - d\nu
\end{eqnarray}
\end{center}

Cuando el sistema se encuentra limitado espacialmente por una superficie o un defecto (bordes, esquinas, líneas, etc.) debe tenerse en cuenta
 una contribución local a la energía libre del sistema $f_{def}$ cuya parte singular presenta el mismo comportamiento que el dado en la ec. \ref{eq:sca_hyp}
 para $f$, en el caso de un defecto con dimensión $d-1$:
 
\begin{equation}
	\label{eq:sca_def}
	f_{def}(t,h_{def},\frac{1}{L})=b^{-(d-1)}f_{def}(b^{1/\nu}t,b^{d-x}h_{def},\frac{b}{L})
\end{equation}
\\
 donde $h_{def}$ representa el campo magnético sobre la superficie o defecto.\\
Repitiendo los argumentos utilizados anteriormente, las relaciones \ref{eq:xdefinition}-\ref{eq:sca_rel_1} se cumplen para
 los correspondientes exponentes de superfice (defecto):
 
\begin{equation}
	\label{eq:sca_rel_surf}
	M_{def}\sim L^{-x_{def}}, \beta_{def} =x_{def}\nu
\end{equation}
\\
La funci\'on de correlaci\'on para un par de spines se obtiene como la derivada funcional de la energ\'ia libre respecto del campo magn\'etico 
 $G(r,t)=\mean{\sigma(0)\sigma(r)}=\frac{\delta F}{\delta h(0) \delta h(r)}$ y por lo tanto su comportamiento ante una
 transformaci\'on de escala viene dado por:
 
\begin{equation}
	\label{eq:sca_corr}
	G(r, t)=b^{-2x}G(\frac{b}{r}, b^{1/\nu}t)
\end{equation}
\\
 en las cercan\'ias del punto cr\'itico, $t\rightarrow 0$, y para una elecci\'on de $b=r$ las correlaciones decaen como
 una ley de potencias:

\begin{equation}
	\label{eq:sca_corr_pot}
	G(r, t=0)=\frac{G_{0}}{r^{2x}}
\end{equation}
\\
En presencia de defectos, pueden analizarse las correlaciones entre spines pertenecientes al defecto, y las
 correlaciones entre estos y los que no pertenecen al defecto.\\
En el caso de un defecto en forma de l\'inea las primeras
 son correlaciones a lo largo de la l\'inea, longitudinales, y las \'ultimas son perpendiculares al defecto. En ambos casos
 el exponente de la funci\'on de correlaci\'on est\'a relacionado con el exponente cr\'itico $x_{def}$ de la magnetizaci\'on
 del defecto definido en \ref{eq:sca_rel_surf}:

\begin{equation}
	\label{eq:sca_corr_pot_def}
	G_{\parallel}(r, t=0)\sim{r^{-2x_{def}}}, \; \; \; \;  G_{\perp}(r, t=0)\sim{r^{-(x+x_{def})}}
\end{equation}
\\

Estas expresiones nos permiten estudiar el comportamiento cr\'itico de la funci\'on de correlaci\'on a partir
 de medidas realizadas para el exponente cr\'itico $x_{def}$ asociado a la magnetizaci\'on del defecto.\\

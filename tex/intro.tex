%%%%%%%%%%%%%%%%%%%%%% Intro %%%%%%%%%%%%%%%%%%%%%%%%%%%%%

\section{Introducci\'on}
\label{sec:intro}
Los modelos bidimensionales han tenido un importante papel en la mec\'anica estad\'istica en relaci\'on al estudio de los
 fen\'omenos cr\'iticos y las transiciones de fase. Entre los más estudiados se encuentran los que representan
 varios tipos de átomos dispuestos en una red vinculados por algún tipo de interacción, por ejemplo el modelo de Ising, Eigth-Vertex (8V)\cite{model_8V},
 o Ashkin-Teller (AT)\cite{ashkin_teller_43}. Este tipo de modelos ha sido objeto de estudios durante las últimas décadas,
 algunos de ellos han sido resueltos exactamente a través de sofisticados métodos teóricos y estudiados mediante varios
 métodos numéricos, por lo que hay abundante información sobre su comportamiento. Sin embargo, la posibilidad de obtener y estudiar en un laboratorio
 materiales bidimensionales \cite{exp_ultrathin_magfilms} ha despertado interés en el estudio de nuevos problemas y nuevos escenarios,
 entre ellos, la extensión de los modelos mencionados para
 reproducir el comportamiento de defectos o impurezas presentes en los materiales.\\
La introducción de interfaces o defectos lineales en modelos bidimensionales \cite{linear_defects2D, interf, interf_AT} ha dado como resultado importantes
 modificaciones en el comportamiento crítico de los sistemas.\\
El comportamiento crítico del modelo de Ashkin-Teller en dos dimensiones con un defecto en forma de línea ha sido estudiado
 recientemente utilizando el método numérico density matrix renormalization (DMRG) en la descripción del modelo en el límite Hamiltoniano \cite{AT_lajko},
 que conduce a una representación cuántica en una red de dimensión menor a la del modelo clásico, así como mediante métodos teóricos, en particular integrales de camino
 en la descripción del modelo en términos de campos fermiónicos\cite{AT_naon}.\\
Las discrepancias entre los resultados obtenidos mediante métodos teóricos y numéricos , respecto a la dependencia de los exponentes críticos con algunos parámetros
 del hamiltoniano son una fuerte motivación para tratar el problema mediante un método diferente.\\
En este trabajo estudiaremos el comportamiento crítico de este modelo utilizando el método de simulaciones computacionales Montecarlo (MC), ampliamente utilizado
 para el estudio de fenómenos críticos en sistemas bidimensionales, directamente sobre el sistema clásico. Nuestro objetivo es arrojar algo de luz sobre los
 resultados actuales.\\
